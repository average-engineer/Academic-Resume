\section{\underline{Berufserfahrung}}

\large{\textbf{Masterand/Werkstudent}}
\hfill
%\hspace{2.9 in}
\large{\textbf{Oktober 2023 - laufende}}\\
\large{\emph{Rheinmetall Technology Centre - New Technologies, Rheinmetall AG, Neuss, Deutschland}}
\begin{itemize}
	\item Entwicklung eines Lokalisierungsalgorithmus für Schwärme unbemmanter Luftfahrzeuge (UAV), die keine GPS-Unterstützung haben, mithilfe Ultra-wideband (UWB)-Sensoren.
	\item Umsetzung des entwickelten Algorithmus im Arducopter, der ein Open-Source Autopilot-Software für Drohnen ist, um das Algorithmus in Flugtests zu verifizieren.
	\item Entwicklung eines Simulationsmodells eines integrierten Systems für ein einzelnes UAV, einschließlich eines Flugreglermodells, eines 6-Freiheitsgraden dynamisches Modells, Sensormodellen von IMU, GPS, Magnetometer, Barometer sowie eines Modells des erweiterten Kalman-Filters für Lage und Geschwindigkeit-Einschätzung.
\end{itemize}

\vspace{0.1 in}

\large{\textbf{Werkstudent}}
\hfill
\large{\textbf{Juni 2022 - September 2023}}\\
\large{\emph{Rheinmetall Technology Centre - New Technologies, Rheinmetall AG, Neuss, Deutschland}}
\begin{itemize}
	\item\large{Entwicklung eines mathematischen Modells für die Kraftauswertung eines flexiblen elektrostatischen Synchronaktuators anhand vom Konzept \emph{Method of Moments}, das ein Dikretisierungstechnik für elektrische Felder ist.}
	\item\large{Wartung des entwickelten Modell-Code-Base mithilfe der Softwareentwicklungprinzipien wie objektorientierte Programmierung, Version-Control und umfassende Code-Dokumentation.}
\end{itemize}

\vspace{0.1 in}

\large{\textbf{Projekt-Assistant}}
\hfill
%\hspace{4.1 in}
\large{\textbf{Oktober 2022 - April 2023}}\\
\emph{\large{Institut für Getriebetechnik, Maschinendynamik und Robotik (IGMR), RWTH Aachen, Deutschland}}
\begin{itemize}
	\item \large{Entwurf eines Simulationsmodells für ein Standard-Mountainbike in Simpack zum Zweck eines \emph{Semi-Analytical} Modells, worin das Mehrkörperdynamischesmodell teilweise von den Belastungen auf dem echten Fahrrad abhängt.}
	\item\large{Aufstellung einer Co-Simulation zwischen Simpack und Simulink, um einen Regelkreis zu simulieren, der das durch die gemessenen Belastungen angeregte Fahrrad-Modell stabilisiert.}
	\item\large{Die Arbeit wurde als ein "Mini-Thesis" erfasst, das \href{https://github.com/average-engineer/MiniThesis_IGMR/blob/master/Thesis.pdf}{\large{\textit{hier}}} zugegriffen werden kann.}
\end{itemize}

\vspace{0.1 in}

\large{\textbf{Research Assistant}}
\hfill
%\hspace{4.1 in}
\large{\textbf{September 2020 - Juli 2021}}\\
\emph{\large{Thapar Institute of Engineering and Technology, Patiala, Indien}}
%\textbf{Dynamic Modelling and Control Design of Augmentative Lower Extremity Exoskeleton}
\begin{itemize}
	\item \large{Dynamische Modellierung eines Kraft-Augmentation-Exoskeletts, der vom \emph{Defence Bio-Engineering and Electro-Medical Laboratory} (DEBEL) in Indien entworfen wurde.}
	\item\large{Reduzierte Ordnung Modellierung des Menschen und des Exoskeletts als ein gekoppeltes Mehrkörpersystem.}
	\item\large{Entwicklung eines \emph{Computed Torque Control} Algorithmus zur Kraftverstärkung.}
	\item\large{Alle Veröffentlichungen aus diesem Projekt werden unter den Abschnitt \emph{Konferenz-Präsentationen und Veröffentlichungen} erwähnt. }
\end{itemize}




