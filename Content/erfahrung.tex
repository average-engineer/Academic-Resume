\section{\underline{Berufserfahrung}}

\large{\textbf{Masterand/Werkstudent}}
\hfill
\large{\textbf{Oktober 2023 - August 2024}}\\
\href{https://www.rheinmetall.com/de/karriere/rheinmetall-als-arbeitgeber/menschen-projekte/karriere-extra-technology-center}{\large{\emph{Rheinmetall Technology Centre - New Technologies, Rheinmetall AG, Neuss, Deutschland}}}
\begin{itemize}
	\item Entwicklung eines Lokalisierungsalgorithmus für Schwärme unbemmanter Luftfahrzeuge (UAV), die keine GPS-Unterstützung haben, mithilfe von Ultra-wideband (UWB)-Sensoren und Umsetzung des entwickelten Algorithmus im Ardupilot, das ein Open-Source Autopilot-Software für Drohnen ist, um das Algorithmus in Flugtests zu verifizieren.
	\item Entwicklung eines Simulationsmodells eines integrierten Systems für ein einzelnes UAV und anschließend ein Schwarm-Netzwerk mit vier Drohnen, einschließlich eines Flugreglermodells, eines 6-Freiheitsgraden dynamisches Modells, Sensormodellen von IMU, GPS, Magnetometer, Barometer sowie eines Modells des erweiterten Kalman-Filters für Lage und Geschwindigkeit-Einschätzung.
\end{itemize}

\vspace{0.1 in}

\large{\textbf{Werkstudent}}
\hfill
\large{\textbf{Juni 2022 - September 2023}}\\
\href{https://www.rheinmetall.com/de/karriere/rheinmetall-als-arbeitgeber/menschen-projekte/karriere-extra-technology-center}{\large{\emph{Rheinmetall Technology Centre - New Technologies, Rheinmetall AG, Neuss, Deutschland}}}
\begin{itemize}
	\item\large{Entwicklung eines mathematischen Modells für die Kraftauswertung eines flexiblen elektrostatischen Synchronaktuators anhand vom Konzept \emph{Method of Moments}, das ein Dikretisierungstechnik für elektrische Felder ist.}
	\item\large{Wartung des entwickelten Modell-Code-Base mithilfe der Softwareentwicklungprinzipien wie objektorientierte Programmierung, Version-Control und umfassende Code-Dokumentation.}
\end{itemize}

\vspace{0.1 in}

\large{\textbf{Projekt-Assistant}}
\hfill
%\hspace{4.1 in}
\large{\textbf{Oktober 2022 - April 2023}}\\
\href{https://www.igmr.rwth-aachen.de/cms/~jkhpl/igmr/}{\large{\emph{Institut für Getriebetechnik, Maschinendynamik und Robotik (IGMR), RWTH Aachen}}}
\begin{itemize}
	\item \large{Entwurf eines virtuellen Regelkreises, um ein mehrkörperdynamischen Modell eines standarden Mountain-Bikes zur Stabilität zu bringen, das von Lasten angeregt wird, die von auf dem echten Fahrrad gerüstete Sensoren gemessen werden.}
	\item\large{Die Arbeit wurde als ein \href{https://average-engineer.github.io/Projects-Website-Ashutosh-Mukherjee/Resources/MiniThesis.pdf}{\large{\textit{Mini-Thesis}}} erfasst.}
\end{itemize}

\vspace{0.1 in}

\large{\textbf{Research Assistant}}
\hfill
\large{\textbf{September 2020 - Juli 2021}}\\
\href{https://www.thapar.edu/}{\large{\emph{Thapar Institute of Engineering and Technology, Patiala, Indien}}}
\begin{itemize}
	\item\large{Dynamische Modellierung eines Kraft-Augmentation-Exoskeletts, der vom \emph{Defence Bio-Engineering and Electro-Medical Laboratory} (DEBEL) in Indien entworfen wurde und Entwicklung eines \emph{Computed Torque Control} Algorithmus zur Kraftverstärkung.}
	\item\large{Alle Veröffentlichungen aus diesem Projekt werden unter den Abschnitt \emph{Konferenz-Präsentationen und Veröffentlichungen} erwähnt.}
\end{itemize}

\ifnum\teach = 1
	\vspace{0.1 in}

	\large{\textbf{Teaching Assistant}}
	\hfill
	%\hspace{4.1 in}
	\large{\textbf{Februar 2021 - Juni 2021}}\\
	\href{https://www.thapar.edu/}{\large{\emph{Thapar Institute of Engineering and Technology, Patiala, Indien}}}
	\begin{itemize}
		\item Übung-Leiter für die Vorlesung \emph{Modern Control of Dynamic Systems}, die grundlegende und fortgeschrittene Themen im Bereich von lineare System-Theorie umfasst, nämlich die Darstellung eines linearen Systems im Zustandsraum, den Entwurf eines Reglers und eines Beobachters anhand von \emph{Pole Placement} und einige Themen in der optimale Regelung wie \emph{LQR} und \emph{LQG}.
		\item Erstellung von Zwischenprüfungen und Hausaufgaben aus der Vorlesung.
	\end{itemize}
\fi



