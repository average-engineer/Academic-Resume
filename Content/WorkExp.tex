% Content added in a chronological order

\section{\underline{Work Experience}}

\large{\textbf{Master Thesis Student/Working Student}}
\hfill
%\hspace{2.9 in}
\masterarbeitdate\\
\href{https://www.rheinmetall.com/de/karriere/rheinmetall-als-arbeitgeber/menschen-projekte/karriere-extra-technology-center}{\large{\emph{Rheinmetall Technology Centre - New Technologies, Rheinmetall AG, Neuss, Germany}}}\\
\large{\textbf{Distributed Data-Fusion and Control over a network of Unmanned Aerial Vehicles}}
%\masterarbeitBetreuer
\begin{itemize}
	\item Development of a localization algorithm for a swarm of GPS-denied UAVs using Ultra-wideband (UWB) sensors and embedding the developed localization algorithm into \emph{Arducopter}, an open-source auto-pilot for UAVs, in order to facilitate verification of the algorithm through flight tests.
    \item Development of a simulation model of an integrated flight control system for a single UAV, including a flight controller, 6 degree-of-freedom UAV dynamic model, IMU, Magnetometer, Barometer and GPS sensor models and an Extended Kalman Filter for attitude estimation.
\end{itemize}

\vspace{0.1 in}

\large{\textbf{Werkstudent (Working Student)}}
\hfill
%\hspace{2.9 in}
\rheinmetalldate\\
\href{https://www.rheinmetall.com/de/karriere/rheinmetall-als-arbeitgeber/menschen-projekte/karriere-extra-technology-center}{\large{\emph{Rheinmetall Technology Centre - New Technologies, Rheinmetall AG, Neuss, Germany}}}
\begin{itemize}
\item\large{Development of a mathematical model for the force analysis of a flexible electrostatic synchronous actuator based on the concept of \emph{Method of Moments}, a discretization technique for electric fields.}
\item\large{Maintainance of the developed mathematical model code-base, making use of software engineering principles like object orientation, version control and comprehensive code documentation}
\end{itemize}

\vspace{0.1 in}

\large{\textbf{Project Assistant}}
\hfill
%\hspace{4.1 in}
\minithesisdate\\
\href{https://www.igmr.rwth-aachen.de/cms/~jkhpl/igmr/}{\large{\emph{Institut für Getriebetechnik, Maschinendynamik und Robotik (IGMR), RWTH Aachen}}}
\begin{itemize}
\item \large{Scripting Multi-Body simulation models of a standard mountain bike in Simpack using the \emph{Semi-Analytical Approach} where the multibody model partly depends on real-time sensing of loads on the actual bike and setting up co-simulation between Simpack and Simulink in a closed loop simulation for stabilizing the multibody model excited by the measured loads. }
%\item\large{Testing various linear control techniques in order to achieve disturbance rejection}.
\item\large{The \href{https://github.com/average-engineer/MiniThesis_IGMR/blob/master/Thesis.pdf}{\large{work}} in this project culminated in a mini-thesis worth 9 ECTS.}
\end{itemize}

\vspace{0.1 in}

\large{\textbf{Research Assistant}}
\hfill
%\hspace{4.1 in}
\drdodate\\
\href{https://www.thapar.edu/}{\large{\emph{Thapar Institute of Engineering and Technology, Patiala, India}}}
%\textbf{Dynamic Modelling and Control Design of Augmentative Lower Extremity Exoskeleton}
\begin{itemize}
\item\large{Reduced-order dynamic modeling of a strength augmentation exoskeleton designed by \href{https://www.drdo.gov.in/drdo/labs-and-establishments/defence-bio-engineering-electro-medical-laboratory-debel}{\large{\emph{Defence Bio-Engineering and Electro-Medical Laboratory}}} (DEBEL), India and development of a \emph{Computed Torque Control} algorithm for strength augmentation.}
\item\large{All published work output from this project can be referred from \emph{Conference Presentations and Publications} section.}
\end{itemize}

\ifnum\teach = 1
	\vspace{0.1 in}
	
	\large{\textbf{Teaching Assistant}}
	\hfill
	%\hspace{4.1 in}
	\thapardate\\
	\href{https://www.thapar.edu/}{\large{\emph{Thapar Institute of Engineering and Technology, Patiala, India}}}
	\begin{itemize}
		\item Teaching Assistant for the graduate level course \emph{Modern Control of Dynamic Systems}, which covers basic and advanced topics in linear systems theory like state-space representations, canonical forms, controller and observer design using pole-placement, and some concepts of optimal control theory.
		\item Helped set mid-term examinations and quizzes, and grade them as part of assitantship duties.
	\end{itemize}
\fi




