% Content added in a chronological order

\section{\underline{Work Experience}}

\large{\textbf{Master Thesis Student}}
\hfill
%\hspace{2.9 in}
\masterarbeitdate\\
\large{\emph{Rheinmetall Technology Centre - New Technologies, Rheinmetall AG, Neuss, Germany}}\\
\large{\textbf{Distributed Data-Fusion and Control over a network of Unmanned Aerial Vehicles}}\\
\masterarbeitBetreuer
\begin{itemize}
    \item Development of a simulation model of an integrated flight control system for a single UAV, including a flight controller, 6 degree-of-freedom UAV dynamic model, IMU, Magnetometer and GPS sensor models and an Extended Kalman Filter for attitude estimation.
    \item Implementation of inertial strapdown algorithms to propogate the attitude and other states of the UAV in time.
    \item Design of an attitude estimation filter based on the attitude error rotation vector dynamics, also known as \emph{Multiplicative Extended Kalman Filter}
    \item Implementation of waypoint navigation algorithms for the UAV flight controller
    \item Design of a vector measurement model for a network of UAVs as a \emph{multilateration problem}.
\end{itemize}

\large{\textbf{Werkstudent (Working Student)}}
\hfill
%\hspace{2.9 in}
\rheinmetalldate\\
\large{\emph{Rheinmetall Technology Centre - New Technologies, Rheinmetall AG, Neuss, Germany}}
\begin{itemize}
\item\large{Involved in a project about a novel electrostatic synchronous actuator for flexible exoskeletons}
\item\large{Developing a mathematical model for the force analysis of the actuator based on the concept of \emph{Method of Moments}, a discretization technique for electric fields.}
\item\large{Validation of the simulation models against real-time testing of the actuator prototype}
\item\large{Maintainance of the developed mathematical model, for using it as an in-house solver, making us of software engineering principles like object orientation, version control and comprehensive code documentation}
\end{itemize}

\vspace{0.1 in}

\large{\textbf{Project Assistant}}
\hfill
%\hspace{4.1 in}
\minithesisdate\\
\emph{\large{Institut für Getriebetechnik, Maschinendynamik und Robotik, RWTH Aachen}}
\begin{itemize}
\item \large{Scripting Multi-Body simulation models of a standard mountain bike in Simpack using the \emph{Semi-Analytical Approach} where the multibody model partly depends on real-time sensing of loads on the actual bike.}
\item\large{Setting up co-simulation between Simpack and Simulink for a closed loop simulation for stabilizing the multibody model of the bike excited by the measured loads.}
\item\large{Testing various control techniques like \emph{Position Feedback} and \emph{Force pre-control} in order to achieve disturbance rejection so that the actual loads on the real bike can be reproduced in the multi-body model}.
\item\large{The work in this project culminated in a mini-thesis worth 9 ECTS, which can be accessed \href{https://github.com/average-engineer/MiniThesis_IGMR/blob/master/Thesis.pdf}{\large{\textit{here}}}.}
\end{itemize}

\vspace{0.1 in}

\large{\textbf{Research Associate}}
\hfill
%\hspace{4.1 in}
\drdodate\\
\emph{\large{Thapar Institute of Engineering and Technology, Patiala, India}}
%\textbf{Dynamic Modelling and Control Design of Augmentative Lower Extremity Exoskeleton}
\begin{itemize}
\item \large{Dynamic modeling of a strength augmentation exoskeleton designed by \emph{Defence Bio-Engineering and Electro-Medical Laboratory} (DEBEL), a branch of \emph{Defence Research and Development Organization} (DRDO), India}
\item\large{Modeling of Human and Lower Extremity Exoskeleton in the form of coupled multi-body systems in which the Human is the master and the exoskeleton is the slave.}
\item\large{Development of a Computed Torque Control algorithm based on control partitioning for Strength Augmentation of the Pilot wearing the Lower Extremity Exoskeleton.}
\item\large{Various sections of the work done for this project are being submitted in various journals and conferences related to Biomechanics and Multi-Body Dynamics, and all published work can be referred from \emph{Conference Presentations and Publications} section.}
\end{itemize}

\begin{comment}
	\vspace{0.1 in}
	
	\large{\textbf{Intern, Order Management and Assembly Department}}
	\hfill
	%\hspace{1.2 in}
	\siemensdate\\
	\large{\emph{Siemens Ltd., Vadodara, India}}
	\begin{itemize}
		\item\large{Developed a solver in C for allocation of jobs (processes) to different machines present in the shop floor in order to optimize the aggregate machining lead times}
		\item\large{Designed an induction heating apparatus for heating of rotor wheel discs of steam turbines}
		\item\large{Increased robustness of fixtures for machining of stator guide blade carriers of steam turbines}
		\item\large{Redesigned and fabricated a machining and blading stand for rotor wheel discs of steam turbines }
	\end{itemize}.
\end{comment}


