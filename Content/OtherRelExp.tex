\section{\underline{Other Relevant Experience}}
\large{\textbf{Undergraduate Thesis Project}}
\hfill
%\hspace{3in}
\majordate
\begin{enumerate}
\item\large{\textbf{Development of a Test Rig for measuring propeller thrust}}
%\hspace{0.3 in}
%\majordaterig
\begin{itemize}
\item \large{Built a test stand acting as an alternative to the wind tunnel for measuring the thrust produced by a propeller mounted on it.}
\item\large{Implemented Arduino Uno controlled circuits for driving the propeller motor using a brushless DC motor and capturing and displaying the speed of the propeller using an IR sensor-based tachometer.}
\end{itemize}

%\vspace{2 in} % Makeshift verticle space added for formatting the content across pages properly

% Adding border to the second page
%\begin{tikzpicture}
%[remember picture, overlay] \draw[line width=2pt] ($(current page.north west)+(0.1in,-0.2in)$) rectangle %($(current page.south east)+(-0.1in,0.2in)$);
%\end{tikzpicture}

\item\large{\textbf{Design and Analysis of a propeller for slow-flying Quad-copters}}
\begin{itemize}
\item\large{Generated and modified propeller designs iteratively based on required flying conditions and propeller thrust using QMIL, a first-order propeller design tool}
\item\large{Used QPROP, a solver for calculating propeller performance to generate propeller efficiency and thrust curves for the designed propellers and reiterated the designing process until a design giving desirable propeller performance was achieved.}
\item\large{Assisted in second-order design validation using computational fluid dynamics (CFD) once the propeller design showed better performance than a market standard propeller.}
\item\large{Developed a solver acting as an alternative to QPROP in MATLAB for calculating the performance characteristics of a propeller based on Blade Element Momentum Theory.}
\end{itemize}
\end{enumerate}
%\vspace{0.1 in}