\section{\underline{Technische Fähigkeiten}}
\vspace{-1.5 em}
\begin{table}[H]
	\centering
	\begin{tabularx}{\textwidth}{|>{\setlength\hsize{1\hsize}\setlength\linewidth{1\hsize}}X|>{\setlength\hsize{1\hsize}\setlength\linewidth{1\hsize}}X|}
		\hline
		&\\[-2 ex]
		\large{\textbf{Programmierung}} & \large{\textbf{Sonstiges}}\\[0.05 in]
		\hline
		&\\[-5 ex]
		\begin{enumerate}
		\item\large{\textbf{Skripting-Sprachen}}
		\begin{itemize}
			\item\large{MATLAB $\&$ Simulink}
			\item\large{Python}
		\end{itemize}
		\item\large{\textbf{Low-Level-Sprachen}}
		\begin{itemize}
			\item\large{C++}
			\item\large{Java}
		\end{itemize}
		\item\large{\textbf{Allgemein}}
		\begin{itemize}
			\item Objektorientiert Programmierung
			\item Versionskontrolle mit Git
		\end{itemize}
	\end{enumerate} & \begin{enumerate}
			\item\large{Zustand-Einschätzung mit dem erweiterten Kalman-Filter}
			\item\large{Strapdown-Algorithmus für inertiale Navigation}
			\item\large{Sensor-Fusion}
			\item\large{Regression und Klassifizierung mit neoronalen Netzwerke}
			\item\large{Mehrkörperdynamik}
			\item\large{Finite-Elemente-Analyse (Statische und Dynamische)}
			\end{enumerate}\\
		\hline
	\end{tabularx}
\end{table}
